
\documentclass[12pt]{article}

%% Language and font encodings
\usepackage[english]{babel}
\usepackage[utf8x]{inputenc}

\usepackage{booktabs}
\usepackage{tabu}
\usepackage[T1]{fontenc}

%% Sets page size and margins
\usepackage[a4paper,top=3cm,bottom=2cm,left=2.6cm,right=2.6cm,marginparwidth=1.75cm]{geometry}
%\usepackage[a4paper]{geometry}
%% Useful packages
\usepackage{amsmath}
\usepackage{graphicx}
%\usepackage{apacite}
\usepackage[colorinlistoftodos]{todonotes}
\usepackage[colorlinks=true, allcolors=blue]{hyperref}

\title{\textbf{Research statement for the Postdoctoral Researchers at the CMS Experiment in the Helsinki Institute of Physics, Finland}}
\author{Muhammad Gul \\ 
\small \texttt{muhammad.gul@cern.ch}} % \\ \small \texttt{kiani@to.infn.it}}
\date{\small \today}

\begin{document}
\maketitle
\section*{A Path to the High Energy Physics}
From the High School, I was keenly interested to solve mathematical and physics problems and at that time I heard from my physics teacher about the Standard Model of Physics and the contribution of the famous Pakistani scientist \textbf{Prof. Dr. Abdus Salam}. Inspiring from such a great work and my pure and deep interest towards physics as a fundamental language of nature led me to the faculty of physics in Quaid-i-Azam University Islamabad where I completed my bachelor and master degree. During my master, I decided to work in the field of High Energy Physics in the \textbf{Abdus Salam Centre for Physics, Islamabad} with the CMS experiment. Further inspiration came to me after the discovery of the Higgs boson on 4 July, 2012, and finally I decided to make my career in the field of HEP and it turned out to be a successful decision.\\
In master thesis, I worked with different generators like Pythia, MadGraph, HEP programs (HDECAY, HIGLU), CMSSW software infrastructure, generator level study of the SM Higgs in the $\gamma\gamma$ final state, study of the SM $t\bar t$ parameters using 7 TeV data collected by CMS and some basics properties of jets (Sphericity, Aplanarity and Circularity). This wide range of study give me a broad view of the CMS detector and the data analysis. Right after completion of my master degree, I offered PhD position in the Department of Physics and Astronomy in UGhent Belgium which I joined in 2014. During my PhD, I involved in heavy Higgs searches, RPC and GEM detectors study.
\section*{Research Statement}
Modern high energy physics experiments are a complicated synthesis of the $theory$ behind an experiment, design and development of the $detector$ to conduct the experiment, $monitoring$ every detail of the extremely complicated detector, and $analysing$ the obtained data. I had a great opportunity to work on the BSM Higgs searches using CMS data at 13 TeV, the study of the CMS RPC and GEM detectors and develop project for operation and control of the CMS RPC detectors installed in the GIF++ facility at CERN.
  
\subsection*{Physics Analysis}
At 13 TeV, LHC delivered more than 80 fb$^{-1}$ integrated luminosity but we didn't see any hint to new physics. Do we need more data or we have to change our search strategy? The first part of the question is directly related to many new physics channels that are statistically limited and their sensitivity increases in proportion to the integrated luminosity. Once we enter the Phase-II data taking period, we can answer to these questions precisely but it's long away. In present situation, the second part of the question needs more attention as we will have $\sim$100 fb$^{-1}$ data in 2018. Since all the discovered particles including SM higgs are searched as access on the continuum background. Can we search for more complex signal shapes compare to the simple resonance, that might be a peak-dip or dip-peak, only dip or only peak or even more complex (nothingness)?.\\
Inspired from such ideas, first time we started a search for heavy scalar (H) or pseudoscalar (A) Higgs bosons, predicted by multiple new physics models (2HDM, MSSM), targeting the 2016 data collected by CMS experiment. The two particles produced by the gluon fusion and exclusive decaying into $t\bar t$ which further decay into SM particles ($gg\rightarrow A/H\rightarrow t\bar t$). At the LHC, $t\bar t$ are copiously produced from the QCD process ($gg\rightarrow t\bar t$), interfere to the signal and produces a peak-dip structure. This makes the search very interesting and challenging.\\
The search targets mass of the heavy Higgs from 400 GeV ($t\bar t$ mass threshold) to 750 GeV in steps of 100 GeV. The heavy scalar (H) or pseudoscalar (A) are produced independently (one of the two coupling modifier g is always zero) via the Yukawa coupling 
\begin{equation}
\mathcal{L}_{Yukawa} \supset \frac{m_{t}}{\nu}(g_{Htt} t\bar t + ig_{Att}t \bar t).
\end{equation}      
where m$_{t}$ is the top quark mass, $\nu\approx$ 246 GeV is the vacuum expectation value and g's are the coupling modifiers. The $t\bar t$ further decays into electrons, muons and jets. In the present version of the analysis, the semileptonic and dileptonic channels of the $t\bar t$ are included. The results are presented as model independent as well as interpreted in the context of two-Higgs doublet models, in particular in the minimal supersymmetric standard model. The results are consistent with the theory prediction which is a great triumph of this analysis. The analysis has been pre-approved where all information can be seen in the \textsc{CADI} \cite{first1}.\\
In this analysis we used the \textsc{MadGraph} generator for our signal generation and include interference effects from SM $t\bar t$ at LO approximation. The signal is further showered and hadronized with the helped of \textsc{Pythia8} program, using CMS Full Simulation to produce final analysis data sets in the form of \textsc{miniaod}. This chain has majority of application in the SM and BSM searches at CMS.\\
We used almost all particles detected by the CMS detector namely electron, muon, jets (b-jets) and applied all the corrections. For the $p_{Z}$ of MET, we used the Analytical Solver \cite{second2}, uses the Top and W mass constrains, instead of Quadratic Solver. The method can be extended to all types of searches that include neutrino(s) to precisely reconstruct it's longitudinal component of momentum.\\
At LHC, most of the analysis channels have QCD multijet as a background. Due to its overwhelming cross section, it is notoriously difficult to describe by means of MC simulation. To estimate the QCD multijet background properly, we used a data-driven method in this analysis. The method can be applied to all type of searches that includes QCD multijet as background.\\ 
The results of this analysis is represented as model dependent using hMSSM. The NNLO and LO cross sections calculated using the \textsc{Sushi} program with combination of \textsc{2HDMC} and \textsc{Top++} where the results are cross checked with \textsc{MadGraph}. The experience I gained from this study can be utilized in the SM and BSM studies as well as in the Higgs Cross Section Working Group (LHCHXSWG) Yellow Report.
\subsection*{BSM Higgs}
After discovery of the Higgs boson in 2012 by CMS and ATLAS with mass 125 GeV, the main mission of the CERN Large Hadron Collider (LHC) now is, to probe the electroweak symmetry breaking mechanism and the search for possible extensions of the Standard Model (SM) of particle physics. Supersymmetry (SUSY) is considered as the most appealing theory as it addresses several shortcomings of the SM, including the problem of the large hierarchy between the Planck and electroweak scales. In the simplest scenario, the Minimal Supersymmetric Standard Model (MSSM), the Higgs sector consists of three neutral Higgs bosons (h, H, A) and a pair of charged Higgs bosons, H$^{\pm}$.\\
It is known that two efficient channels, $t\rightarrow bH^{+}\rightarrow b\tau\nu$ with low higgs mass and $tan\beta \geq$ 1 and $H/A\rightarrow\tau\tau$ with large rate for high $tan\beta$, can be used to directly search for the heavier MSSM Higgs particles at the LHC and probe part of the [tan$\beta$, M$_{A}$] parameter space. At low mass and high $tan\beta$, the $H/A\rightarrow b\bar b$ has higher rate while for high mass and low $tan\beta$, the $H/A\rightarrow t\bar t$ channel becomes the dominant. The later has explained in detail in the above section as my PhD analysis. H/A/H$^{\pm}$ decay into a number of others channels where the detail can be found here \cite{third3}. \\
The work I did during my PhD is practically applicable to all searches and I am well equipped to own any analysis.  
\subsection*{Detector Study}  
GIF++ is the new CERN irradiation facility for the performance and stability study of the currently installed and future detectors in a high radiation environment produced by HL-LHC machine (3000 fb$^{-1}$). The CMS muon community installed CMS RPC detectors for efficiency and longevity study in 2015 in the GIF++. An R\& D of different gases mixture is under study at GIF++ to replace the two gases C$_{2}$H$_{2}$F$_{4}$ and SF$_{6}$ by eco-friendly gases. My contribution in the GIF++ facility is to make a detector operating and controlling project (commonly known as DCS) based on WinCC-OA. The system controls high voltage and low voltage supplies and monitors temperature, pressure and humidity of both the RPC gas and the environment. The $\gamma$-rays source status and attenuator (to control the flux of $\gamma$-rays) values are accessed through the Data Interchange Protocol (DIP), published centrally by the Engineering Department. All relevant parameters are archived in a Structured Query Language (SQL) database (DB) for offline analysis. The project presented in \texttt{13th Workshop on Resistive Plate Chambers and Related Detectors (RPC2016)} and published \cite{fourth4}. WinCC-OA has been used almost all of the LHC experiments for operation and controlling. I am also participating in the GEM assembly at UGhent where 30 detectors will be assembled at this site. Based on this study I am full confident that I will work on the R\&D of different detectors used in HEP, designing DCS for new experiments, upgrade the existing DCS project in the current experiments to embed new detectors and work on the data base for the data storage.  
\section*{Future Plan}
In my future plan, I like to be part of the High Energy Physics community where I will participate in variety of fields, starting from the design of new detector, performance and up-gradation study of the established one as well as data analysis in different search areas. For the data analysis, I extremely interested to study the BSM Higgs in different channels, using Multi Variate Analysis (MVA) techniques and exploring different observables to discriminate between signal and background processes. My further interest is in the field of top quark physics to understand it's properties both in single top and $t\bar t$ pair. The mismatched between the data and MC in the top $p_{T}$ is also interesting to search for new physics. My interest will go to study jets in detail as we will face extremely large background in HL-LHC era.\\   
My experience in the DCS development and skills acquired during my work on RPC and GEM detectors, makes me confident that I can be a valuable asset in designing and developing challenging projects for high energy physics experiments.\\
In my future plan, I would like to take responsibility as a leading scientist from Pakistan side in the HEP community. My plan extended further to establish a local HEP experiment in Pakistan where new generation can be prepared and equipped with proper knowledge of the HEP for exploring the new horizon in particle physics.\\
I am very interested in working with your group and am confident that I can contribute to the efforts of the group.\\
\begin{thebibliography}{9}
\bibitem{first1} 
A. Popov et al. 
\textbf{Search for $H\rightarrow t \bar t$ with 2016 dataset},
\texttt{http://cms.cern.ch/iCMS/analysisadmin/cadilines?line=HIG-17-027}

\bibitem{second2} 
B. Betchart et al., 
\textbf{Analytic solutions for neutrino momenta in decay of top quarks},
\texttt{arXiv:1305.1878v2}
 
\bibitem{third3} 
A. Djouadi at al.,
\textbf{Fully covering the MSSM Higgs sector at the LHC},
\\\texttt{arXiv:1502.05653v2}

\bibitem{fourth4} 
Muhammad Gul et al.,
\textbf{Detector control system and efficiency performance for CMS RPC at GIF++},
\\\texttt{JINST 11 C10013}
\end{thebibliography}

\end{document}











