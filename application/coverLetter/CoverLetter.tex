\documentclass[]{article}
\usepackage{lineno}
\usepackage[ngerman]{babel}
\usepackage[utf8]{inputenc}
\usepackage[T1]{fontenc}

\title{\textbf{
Postdoctoral Researchers at the CMS Experiment in the Helsinki Institute of Physics, Finland
}}
\date{}



\setlength{\topmargin}{-5mm}
\setlength{\textwidth}{7in}
\setlength{\oddsidemargin}{-8mm}
\setlength{\textheight}{9in}
\setlength{\footskip}{1in}

\begin{document}
\pagestyle{empty}
%\linenumbers
\fontsize{12}{15}
\selectfont
\maketitle
\thispagestyle{empty}
\begin{flushright}
January 24, 2018
\end{flushright} 
\noindent \textbf{Recruitment Committee}\\
\noindent Helsinki Institute of Physics,\\ 
\noindent Finland\\

\noindent Dear Committee Members,
\\
\\
It came to my knowledge through the INSPIRE HEP that you have a vacancy for the Postdoctorial Researcher (CMS Experiment) in the Helsinki Institute of Physics, Finland. I hereby extend my application for the said position within your institute. I am in the last year of my PhD in the University of Ghent, Belgium in the CMS group working on heavy Higgs searches, RPCs and GEM detectors. It would be a tremendous opportunity to continue my research in high energy physics with CMS experiment after my PhD studies.\\

\noindent My PhD studies laid an excellence foundation in understanding physics in general and High Energy Physics in particular. It has helped me acquire a great deal of knowledge in CMS experiment in various fields, Higgs, top quark, jets, leptons, missing transverse energy, generators study like Madgraph, event simulation (Fast and Full simulation), data-driven estimation of background, BSM higgs cross section programs like SusHi, the RPCs and GEM detectors and its control system based on WinCC-OA.\\
 
\noindent During my PhD I involved in the heavy Higgs search in the $t\bar t$ channel ($gg\rightarrow A/H\rightarrow t\bar t \rightarrow$ SM particles). It's the first analysis in the history of CMS for heavy Higgs in this channel which also includes the interference effects from the SM $t\bar t$. The search targets $\mathcal{CP}$-even and odd states of the heavy Higgs and covers mass range from 400 to 750 GeV in the step of 100 GeV. Interestingly the results are compatible with the theory predictions and the analysis is already pre-approved (CADI: \texttt{HIG-17-027}).\\

\noindent I also designed the control and operational system for CMS RPCs detectors installed in GIF++ (Gamma Irradiation Facility) for upgrade testing. The system based on WinCC-OA and has variety of functionalities, operating and controlling the detectors and archiving the data on database. It has been published in \texttt{JINST 11 C10013}. I am also taking part in GEM detector assembly at Ghent site where 30 detectors will be assembled.\\

\noindent In long term I want to be a part of High Energy Physics community and will contribute to HL-LHC, FCC. Pakistan became a member state of CERN recently which is a proud for me as a Pakistani and in future I want to be a member of Pakistani team. We can also make a next generation collider locally in Pakistan once we have enough resources.  \\   

\noindent Upon being given an opportunity to serve your institute, I pledge to utilize my technical skills, mentioned above, coupled with essential transferable skills that I possess in order to aid your CMS team to achieve the organizational goals, thereby contributing to the institution’s development.\\

\noindent I am now based in Belgium, if you wish to contact me in order to discuss my application in details, then you may feel free to contact me any time between 9a.m. and 7p.m. during the week.\\
\noindent I hope you consider my application favourably and provide me with an opportunity to explore my skills, while contributing to your organization's growth. 
\\
\\
Sincerely,\\
\\
Muhammad Gul\\

\noindent \textbf{Enclosure:} Curriculum Vitae, Statement of Research Interests 

\end{document}
