\graphicspath{{chapt_dutch/}{intro/}{chapt2/}{chapt3/}{chapt4/}{chapt5/}{chapt6/}{chapt7/}}
\renewcommand{\thesection}{\arabic{section}}    % chapter without number, so don't use chapterno.sectionno

\renewcommand{\bibname}{References}
% Header
\renewcommand\evenpagerightmark{{\scshape\small Chapter 8}}
\renewcommand\oddpageleftmark{{\scshape\small Summary and Conclusion}}

\chapter[Summary and Conclusion]%
{Summary and Conclusion}\label{summary_conclusion}

\hyphenation{}
Particle physics aims to describe the ultimate structure of matter at the level of its smallest constituents and the interactions among them. 
The Standard Model (SM) of particle physics describes all fundamental particles – leptons (electron, muon, tau, and their corresponding neutrinos), quarks (up, down, charm, strange, top, and bottom), gauge bosons (W$^{\pm}$, Z, $\gamma$), and a Higgs boson. 
The SM includes the interactions (electromagnetic, weak, and strong) and successfully explains how they act on matter particles by exchanging gauge bosons. However, the most familiar force in our everyday lives, gravity, is not part of the Standard Model.
The SM is based on quantum field theory that was developed over the past sixty years and has been tested successfully by many different collider and non-collider experiments. Until recently, the last missing piece of the SM was the Higgs boson that was finally discovered in 2012 by the CMS and ATLAS experiments at the CERN Large Hadron Collider. This discovery represented the ultimate victory of the SM with the experimental verification of the mechanism of spontaneous electroweak symmetry breaking. 

Particle detectors and accelerators have allowed for rapid progress in experimental particle physics research. The technology involved has evolved enormously over the last century, from the early Geiger counters and cloud chambers to today's most powerful Large Hadron Collider (LHC) accelerator and its state-of-the-art particle detectors at the European Laboratory for Particle Physics (CERN) in Geneva. The LHC began operation in 2009, and in spring 2010, the energy reached 3.5\,TeV per beam, delivering 4.8\,fb$^{-1}$ by the end of 2011. In 2012, the per-beam energy increased to 4.0\,TeV with an instantaneous luminosity of 6.5$\times 10^{33}\text{cm}^{-2}\text{s}^{-1}$. The machine delivered 23\,fb$^{-1}$ till the end of 2012, and this era is named as Run-I. In early 2013, the CERN accelerator complex shut down for two years of planned maintenance and consolidation, known as the first LHC Long Shutdown (LS1). Finally, the machine began its operations once again in 2015 at a record center-of-mass energy of 13\,TeV and nominal instantaneous luminosity 5$\times 10^{33}\text{cm}^{-2}\text{s}^{-1}$ using 50\,ns bunch spacing. The bunch spacing decreased to 25\,ns and the LHC machine delivered more than 80\,fb$^{-1}$ in 2016–17 successfully. The LHC will operate in 2018 to complete Run-II (2015-2018) with an expected integrated luminosity of more than 100\,fb$^{-1}$. It will be followed by another Long Shutdown (LS2) in 2019–20 with essential upgrades to the LHC and the experiments. Run-III will span 2021–23 with an expected delivered luminosity of 300\,fb$^{-1}$, which will be followed by the third Long Shutdown (LS3), allowing final upgrades of the machine and experiments before the High Luminosity LHC (HL-LHC) program. Starting in 2026, the HL-LHC is expected to provide an instantaneous luminosity of 5$\times 10^{34}\text{cm}^{-2}\text{s}^{-1}$ with a potential peak value  of 2$\times 10^{35}\text{cm}^{-2}\text{s}^{-1}$ at the beginning of fills. 

The HL-LHC machine will induce a higher background radiation as compared to the present operating conditions, which will challenge the detectors. It is important to study the performance and stability of currently installed and future detectors in a high-radiation environment. With a focus on these requirements, the CERN Engineering- (EN) and Physics Department (PH) created a joint project – Gamma Irradiation Facility (GIF++), which is a new irradiation facility located north of the CERN Super Proton Synchrotron (SPS). It is a unique place where high energy ($\approx$ 100\,GeV) charged particles (mainly muons) are combined with a high flux of gamma radiation (662\,keV) produced by a 13.9\,TBq $^{137}$Cs source. It allows the cumulation of doses equivalent to HL-LHC experimental conditions within a reasonable time period.

The Gent group has been working on CMS RPCs since 2007 and is also involved in the upgrade of the CMS RPCs system. At HL-LHC, RPCs and associated electronics will operate at a much higher luminosity that will produce ageing effects. In order to study the ageing effects of the RPCs and its electronics for the HL-LHC upgrade program, the CMS RPC community installed several different types of RPC detectors at the GIF++. A dedicated Detector Control System (DCS) has been developed using the \textsc{WinCC-OA} tool to control and monitor these detectors and store the measured parameters’ data. \textsc{WinCC-OA} is an open architecture adopted by all complicated experiments at the LHC and relies consistently on object orientation to process images and the database structure. This enables efficient and simple mass engineering and swift creation of projects with a number of parallel developments. The performance of the CMS RPC chambers at GIF++ has been studied and compared at different radiation levels. At a rate of 600\,Hz-cm$^{-2}$, i.e. the maximum rate expected at the HL-LHC, the maximum efficiency of the chamber is 95\%. The detector performance is stable at 34\% of the total accumulated charge expected at the HL-LHC.

The discovery of the Higgs boson (spin-zero resonance) with a mass of 125\,GeV by the CMS and ATLAS experiments at the Large Hadron Collider (LHC) completes the SM of particle physics. However, it doesn't eliminate the possibility that additional Higgs states with masses above or below 1\,TeV exist.  
Higgs mass hierarchy is one of the well-known motivations for considering the existence of an extended Higgs sector in SM extensions, such as the MSSM and 2HDM.
These models predict more than one Higgs state; for example, in the MSSM, there are five Higgs states – two of them are charged (H$^{\pm}$) and three are neutral with one SM like (h), one heavy scalar (H), and one pseudoscalar (A). The fact that the properties of the discovered particle seem to be compatible with the expectations of the SM demands that any such model reproduces the SM Higgs boson, which is usually identified using the lightest state available. Such a configuration is referred to as the alignment limit.

The top quark is one of the heaviest particles in SM and is produced mainly in a pair at the LHC. It decays exclusively into a W boson and a b quark, where the W boson further decays into two jets or leptons and its neutrino. The decay mode (leptonic or hadronic) of the W bosons from the two top quark characterizes the top pair decay channels as semileptonic, dileptonic, or fully hadronic.  
Its large mass naturally leads to a strong coupling between the top quark and the additional Higgs bosons. As a result, the decay into a pair of top quarks can have a large branching ratio when kinematically allowed.
This decay channel is especially favourable for a $\mathcal{CP}$-odd state, which cannot couple to the weak bosons.
However, in the 2HDM model, couplings of the heavy $\mathcal{CP}$-even state to W and Z bosons also vanish in the alignment limit; in the MSSM, they are suppressed when $m_A \gg m_Z$ (the decoupling limit).
Owing to this and other factors, various theories beyond the standard model (BSM) can manifest themselves in the $t\bar t$ final state.
Studying it experimentally, however, is a challenging task because of the large SM $t\bar t$ background and the need to reconstruct decays of the top quarks with high precision.
A number of searches for resonances in the m$_{t\bar t}$ spectrum have been performed, but no positive results have been obtained so far.
An important feature of the $pp \rightarrow t\bar t$ process in the presence of a heavy Higgs boson is the interference with the SM continuous background.
Because of a non-trivial phase in the involved loop-induced coupling of the Higgs boson to gluons, the m$_{t\bar t}$ lineshape can show a peak–dip, dip–peak, or dip-only structure instead of the usual resonance peak.
The spectrum of possible deviations becomes even richer when one includes BSM particles in the loop, such as top squarks, or allows for a violation of the $\mathcal{CP}$~symmetry in the Higgs sector.
It should be noted that even in the limit of a narrow resonance, the impact of the interference is not negligible as it changes the size of the peak.

The analysis presented in this thesis focuses on the search for a heavy Higgs boson decaying into a pair of top quarks in the semileptonic final state, using around 36 fb$^{-1}$ of pp collision data collected in 2016. The selected events contain a muon or an electron and at least four jets – at least two of which must be b-tagged.
The search targets two pure $\mathcal{CP}$ states with masses 400, 500, 600, and 750\,GeV, where each mass point has five different widths (2.5, 5, 10, 25, 50)\%. In total, 40 signal samples ($2\times 4\times 5$) for resonance and 40 for interference are produced. The interference with the QCD $t\bar t$ production is found to play an important role and is explicitly taken into account. 

The production of interference samples involves the modification of Fortran code generated by $\textsc{MadGraph}$\\
$\textsc{5\_a-mc@nlo}$.
Let $C_{\Phi tt}$ be the coupling of the heavy Higgs boson to top quarks, which is utilized by the routine that evaluates the squared matrix element $\abs{\mathcal{M}}^2$.
The code is modified at the point where this routine is called, in the following manner.
First, $\abs{\mathcal{M}}^2$ is evaluated for the nominal value of $C_{\Phi tt}$ and saved.
Then, the $C_{\Phi tt}$ sign is flipped and $\abs{\mathcal{M}}^2$ is computed again, yielding the value of $\abs{\mathcal{M}(-C_{\Phi tt})}^2$.
The effective coupling of $\Phi$ to gluons is controlled by an independent parameter in the routine and therefore is not affected by the modification of $C_{\Phi tt}$.
As a result, the interference terms in $\abs{\mathcal{M}}^2$ flip the sign, while the SM terms (independent of the $\Phi$~couplings) and the resonant BSM part (proportional to $C_{\Phi tt}^2$) are left unchanged.
Finally, the value of $(\mathcal{M}^{2}(C_{\Phi tt}) - \mathcal{M}^{2}(-C_{\Phi tt})) / 2$ is computed and used replacing the original squared matrix element everywhere.
Everything but the interference terms are cancelled out in this computation.

The $t\bar t$~system is reconstructed by utilizing kinematic constraints imposed by the known masses of the top quark and the W~boson. The dominant background is $t\bar t$~production, which is modelled with \textsc{Powheg~v2} and $\textsc{Pythia~8}$. The remaining minor backgrounds are modelled as single top tw-channel with \textsc{Powheg} and $\textsc{Pythia~8}$; single top t-channel with \textsc{Powheg~v2} and $\textsc{Madspin}$; single top s-channel and TTZ with \textsc{Amcatnlo} and $\textsc{Pythia~8}$; wjets and Drell-Yan with \textsc{Madgraph} and $\textsc{Pythia~8}$; di-boson with $\textsc{Pythia~8}$; TTZ and TTWjets with \textsc{Amcatnlo}; and \textsc{Madspin} and $\textsc{Pythia~8}$. 

The multijet background is impossible to model from simulation and therefore has been determined from data using data-driven techniques. The number of events from the multijet background is estimated using the so-called ABCD method. The method is applied for the $M_{T}^W$ variable~\ref{Eq:MtW} and the lepton's relative isolation~$I$. The shape of the (two-dimensional) distribution of the search variables for the multijet background is modelled from a region with an inverted selection on the lepton's relative isolation~$I$. The expected contribution from backgrounds that produce prompt leptons is subtracted from the observed data; the resulting distribution is then attributed to the multijet background.

Corrections to simulations, i.e pile-up re-weighting, jet energy scale and resolution, b-tagging, triggers and lepton efficiencies, and generator level weights are applied using CMS standard procedures. Top $p_{T}$ correction is applied using an empirical reweighting based on the observed $p_{T}$~distributions of top quarks. The empirical procedure is further compared with the theory prediction; top $p_{T}$ uncertainty is considered. One of the dominant uncertainties in this analysis originates from the jet calibration. This is evaluated by varying the multiplicative jet energy correction within its uncertainties. Jet Energy Resolution (JER) data-to-simulation scale factors are also simultaneously varied within their uncertainties for all jets. Uncertainties related to b-tagging are evaluated by varying the values of the scale factors within the respective uncertainties. Uncertainties related to trigger scale factors and scale factors of lepton identification are taken into account. 
Integrated-luminosity uncertainty is considered as 2.5\%. The normalization of the data-driven template for the multijet background is varied conservatively by ${+100}/{-50}$\%, independently in the muon and electron channels. Renormalization and factorization scale ($\mu_{R}$ and $\mu_{F}$) uncertainties are taken from the independent variations of the two scales by a factor of two in each direction. Uncertainties related to Initial and Final State Radiation (ISR, FSR) and Underlying Event (UE) are taken from dedicated samples. For top quark mass uncertainties, two simulated samples of $t\bar t$ are used with top quark mass $\text{m}_t$ = 169.5 and 175.5\,GeV. The $h_{\text{damp}}$ parameter in \textsc{POWHEG}, which controls the suppression of radiation of additional high-$p_{T}$ jets, is changed from its nominal value of 1.58$m_{t}$ to 0.99$m_{t}$ and 2.24$m_{t}$. Statistical uncertainties on the dominant $t\bar t$ background are taken into account. The variations in ISR and UE do not produce statistically significant changes in the shape of the distribution and are consequently dropped. The Probability Density Functions (PDF) uncertainties are computed from the set NNPDF3.0. They are described by 100 alternative versions of the PDF, referred to as ``MC replicas'', which are provided in the form of dedicated event weights. 

To probe for the presence of the hypothesized $\Phi$(H/A)~particle, a two-dimensional distribution of the mass of the reconstructed $t\bar t$~system, m$_{t\bar t}$ and an angular variable~$\abs{\cos\theta^*}$ that reflects the spin of the $s$-channel resonance are investigated. The first variable is the reconstructed invariant mass of the $t\bar t$~system, m$_{t\bar t}$, which serves as a proxy for the mass of the heavy Higgs boson. The second variable is $\abs{\cos\theta^*}$, where $\theta^*$ is the angle between the three-momentum of the leptonically decaying top quark in the $t\bar{t}$ rest frame and the three-momentum of the $t\bar{t}$~system in the lab frame.

For the statistical analysis, the Higgs combine tool, based on RooFit/RooStat-based software tools, is used. RooFit is a toolkit integrated with ROOT, which performs different kinds of fits and toy Monte Carlo generations based on user-defined models of the PDF. The interference template, the 2D distribution of the m$_{t\bar t}$, and $\abs{\cos\theta^*}$, has both positive- and negative-weighted events. Due to a limitation of the Higgs combine tool, which can only handle non-negative templates, and for the purpose of interpolation between reference values of $m_\Phi$, the interference template is split into two parts according to the sign of the weight of the contributing events.
The sign of one of the templates is inverted so that both are non-negative.
The two templates are scaled in the model by factors $g^2$ and $-g^2$ respectively ($g$ is the coupling scale factor and for interference, it is proportional to $g^2$), thus reproducing the desired distribution.
With no indication of the presence of the BSM signal, upper limits on the coupling scale factors are computed for various signal hypotheses in the semileptonic channel.

To include the interference effect, leading order (LO) \textsc{Madgraph} is used for signal generation; thereafter, the signal samples are scaled for higher order (NNLO) cross sections. We use the $\textsc{SusHi}$ program interfaced with a $\textsc{2HDMC}$ calculator to calculate scale factors for the resonance part using the hMSSM context; for the interference part, the $\textsc{Top++}$ calculator is used for performing SM $t\bar t$ cross-section calculation. The k-factor is calculated as the ratio of NNLO to LO cross section and applied to the signal cross sections calculated by $\textsc{Madgraph}$. The interference k-factor is taken as the geometric mean of the resonance and the SM $t\bar t$ k-factors. For a single mass point, the k-factors show a constant behaviour for all the considered width hypotheses.

The presented search is sensitive to the high mass of $\Phi$ and low $\tan\beta$, where hMSSM is proposed as it takes the existence of the 125\,GeV Higgs boson directly into account.  
In the case of the hMSSM, it is impossible to fully cover the relevant part of the $(m_A, \tan\beta)$ spectrum with a limited set of samples in terms of fixed masses and widths, such as the ones we produced for this analysis. We used the $\textsc{Roomomentmorph}$ algorithm with which the distributions are interpolated between different masses and widths. The distributions are also extrapolated to lower widths within the narrow-width approximation.

The semileptonic final state has been combined with the dileptonic channel and upper limits are placed on the coupling modifier of these additional Higgs bosons to top quarks using the independent and combined channels. In addition, the results are interpreted in the context of the hMSSM in the $(m_A, \tan\beta)$ plane for the combined channels. The results are compatible with the theory predictions within systematic uncertainties. This analysis is sensitive in the low $\tan\beta$ region and has complementary sensitivity to previous searches, which are either sensitive at high $\tan\beta$ ($2\tau$ final state) or for masses below the $t\bar t$ production threshold~\footnote{In particular, searches for di-Higgs boson production and pseudoscalars decaying to the 125\,GeV Higgs boson and a $Z$~boson, and constraints from the observed couplings of the 125\,GeV Higgs boson}.

So far, no new physics has been found and the results are consistent with the SM theory prediction within systematic uncertainties. The channel could benefit from higher statistics and precise measurement of the $t\bar t$ mass. Using full Run-II data will further increase the sensitivity in this channel while at 300\,fb$^{-1}$ and 3.0\,ab$^{-1}$ – data collected during Run-III and HL-LHC respectively - the $gg\rightarrow H\rightarrow t\bar t$ channel will play a pivotal role in heavy Higgs research. Because of the same final state topology, SM $t\bar t$ is the major background in this channel. To increase the sensitivity, we need to properly reduce the SM $t\bar t$ background using more signal-to-background discriminants. This will be done by using MVA techniques and properly taking into account the negative weights associated with the interference. The k-factors for interference are calculated as the geometric mean of the resonance and SM $t\bar t$ k-factors. In the next iteration, our aim is to calculate NLO k-factors for interference in a manner such that the line shape of the k-factors includes the interference effect. Different signal line shapes, such as pure dip, bump, or even nothingness, will also help to explore this channel.  



\clearpage{\pagestyle{empty}\cleardoublepage}

\renewcommand*{\thesection}{\thechapter.\arabic{section}}       % reset again to chaptnum.sectnum


