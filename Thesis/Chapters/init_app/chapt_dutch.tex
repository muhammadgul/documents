\graphicspath{{chapt_dutch/}{intro/}{chapt2/}{chapt3/}{chapt4/}{chapt5/}{chapt6/}{chapt7/}}
\renewcommand{\thesection}{\arabic{section}}    % chapter without number, so don't use chapterno.sectionno

%\renewcommand{\bibname}{Referenties}
% Header
\renewcommand\evenpagerightmark{{\scshape\small Chapter 9}}
\renewcommand\oddpageleftmark{{\scshape\small Nederlandstalige samenvatting}}

\chapter[Nederlandstalige samenvatting]%
{Nederlandstalige samenvatting}

\hyphenation{}
\def\hyph{-\penalty0\hskip0pt\relax}

Binnen de deeltjesfysica worden de kleinste bouwstenen van materie evenals hun
onderlinge interacties bestude-erd. % hypen used to split the words for line adjustment
Het Standaardmodel (SM) van de
deeltjesfysica beschrijft alle fundamentele deeltjes die we tot op heden
kennen: de leptonen (electron, muon, tau en hun respectievelijke neutrino's),
de quarks (up, down, charm, strange, bottom en top), de ijkbosonon (W$^\pm$,
Z, $\gamma$) en het Higgs boson. Alle zichtbare
materie in het Universum is uit quarks en leptonen uitgebouwd, terwijl de
interacties tussen deze deeltjes succesvol in het SM beschreven wordt door middel van
uitwisseling van ijkbosonen.  
Het SM omvat de electromagnetische,
zwakke en sterke wisselwerking, maar geeft geen beschrijving van de meest
gekende wisselwerking uit ons dagelijks leven, nl. de zwaartekracht.  
Het SM is gebaseerd op een
kwantumveldentheorie die tijdens de voorbije 60 jaar ontwikkeld werd. Dit
model werd gedurende al die jaren succesvol getoetst aan de resultaten van
talrijke experimenten, zowel aan 
deeltesversnellers als daarbuiten. 
Tot voor kort was het laatste ontbrekende stukje van het SM het
zogenaamde Higgs boson, waarvan het bestaan uiteindelijk 
bevestigd werd in 2012 door de CMS en ATLAS
experimenten aan de Large Hadron Collider in het CERN. Deze ontdekking betekende
het ultieme succes van het SM met de experimentele  
verificatie van het mechanisme voor spontane electrozwakke symmetriebreking. 

De opkomst en ontwikkeling van de deeltjesversnellers en -detectoren zorgde voor een enorme
vooruitgang in de experimentele deeltjesfysica. Gedurdende de voorbije eeuw
evolueerde de gebruikte technologie uitermate snel, gaande van de eerste geigertellers en
nevelkamers tot de hedendaagse ultrakrachtige Large Hadron Collider (LHC) en 
bijhorende ultra geavanceerde 
detectoren in het CERN, het Europese laboratorium voor deeltjesfysica te
Gen\`eve in Zwitserland. De LHC werd in 2009 opgestart en in de lente van 2010 werd reeds een
energie van 3.5\,TeV per protonenbundel gehaald, d.w.z.~een totale botsingsenergie
van 7\,TeV, waarmee tegen eind 2011 een ge\"integreerde luminositeit
van 4.8\,fb$^{-1}$ aan de experimenten geleverd werd. 
In 2013 werd de bundelenergie verhoogd naar
4\,TeV met een ogenblikkelijke luminositeit van
$6.5\cdot10^{33}$\,cm$^{-2}$s$^{-1}$, waarbij tegen eind 2013, op het einde van LHC 
Run-I 23\,fb$^{-1}$ afgeleverd werd. Na Run-I werd er een
periode van 2 jaar ingelast voor onderhoud en opwaarderingen van de experimenten (Long shutdown
1, LS1). 
Toen de LHC bij de start van Run-II in 2015 opnieuw in werking trad, werd voor
het eerst een record 
massamiddelpuntsenergie 
van 13\,TeV bereikt. 
De ogenblikkelijke luminositeit bedroeg $5\cdot10^{33}$\,cm$^{-2}$s$^{-1}$, waarbij
de tijd tussen opeenvolgende protonpakketjes in de LHC deeltjesbundels  
50\,ns was. Deze tijd werd verlaagd naar 25\,ns in 2016-2017
waarmee een totale luminositeit van 80\,fb$^{-1}$ behaald werd. De momenteel
aan de gang zijnde Run-II loopt nog 
door tot eind 2018, waarbij een totale luminositeit van meer dan
100\,fb$^{-1}$ verwacht wordt. Tijdens de volgende, tweede lange periode
waarbij de LHC in 2019-2020 stilgelegd zal worden (LS2), zullen er belangrijke
opwaarderingen zowel aan de
versneller als aan de experimenten gebeuren (fase-I upgrade). 
Vervolgens zal LHC Run-III starten in 2021 en duren tot 2023, waarbij er een
ge\"integreerde luminositeit van 
300\,fb$^{-1}$ verwacht wordt. Hierna zal de LHC opnieuw afgezet
worden voor een tweede fase van upgrades om tot de zogenaamde High Luminosity
fase van het LHC programma
(HL-LHC) te komen. De HL-LHC die in 2023 van start zal gaan zou
een ogenblikkelijke luminositeit van $5\cdot10^{34}$\,cm$^{-2}$s$^{-1}$ met pieken
tot $2\cdot10^{35}$\,cm$^{-2}$s$^{-1}$ bereiken.    

De hogere luminositeit van de HL-LHC in vergelijking met de huidige LHC condities
vormt een uitdaging voor de detectoren die blootgesteld zullen worden aan hogere
hoeveelheden straling. Het is daarom dan ook belangrijk om het functioneren en de
stabiliteit van de detectoren te bestuderen onder dergelijke extreme
omstandigheden. Dit kan o.a. gedaan worden aan de zogenaamde Gamma Irradiation Facility (GIF++),
ingericht in het CERN, in het noordelijke deel van het Super Proton
Synchrotron (SPS). In de GIF++ kunnen detectoren blootgesteld worden aan een
bundel van hoogenergetische ($\approx$100\,GeV) geladen deeltjes,
voornamelijk muonen, in combinatie met een achtergrondstraling bestaande uit 
fotonen met energie 662\,keV (afkomstig
van een 13.9\,TBq $^{137}$Cs bron). Hiermee kan men de omstandigheden nabootsen
waaraan de detectoren blootgesteld zullen worden tijdens de HL-LHC fase. 
De Gentse onderzoeksgroep Experimentele Deeltjesfysica is sinds 2007 actief
betrokken bij de ontwikkeling en opwaardering van de zogenaamde Resistive
Plate Chambers (RPCs) in het CMS Muon systeem. Om de kwaliteit alsook het
verouderingsproces van deze RPCs zoals verwacht wordt tijdens de HL-LHC te
testen, worden deze onderzocht in de 
GIF++. Tijdens dit doctoraatswerk werd een controlesysteem (Dedicated Control
System, DCS)
gebruik makende van WINCC-OA ontwikkeld voor de CMS RPC opstelling in de GIF++.
Dit systeem laat toe om op een efficiente en snelle manier alle
detectoren vanop afstand te besturen en op te volgen. Het
gedrag van CMS RPCs, zowel bestaande als nieuw ontworpen detectoren voor de HL-LHC, werd op
deze manier bestudeerd, waarbij een vergelijking van de detectorprestaties bij
verschillende stralingsniveau's gemaakt werd. Bij een deeltjesflux van
600\,Hz/cm$^{2}$ haalden de RPCs nog steeds een
maximum detectorefficientie van 95~\%. Met een tot op heden
geabsorbeerde stralingsdosis
die overeenkomt met 34~\% van de totale dosis verwacht met de HL-LHC, blijven
de detectorprestaties blijven nog steeds stabiel. 
 	
De ontdekking door de CMS en ATLAS experimenten aan het LHC van het
Higgsdeeltje, een boson met spin 0 en een massa van 
$\approx$125\,GeV, zorgde voor de
ultieme bevestiging van de geldigheid van het 
Standaardmodel van de deeltjesfysica. Het is echter niet uitgesloten dat er
naast dat ene Higgsdeeltje
nog meer, zwaardere Higssbosonen bestaan. De extensie van de Higgs sector is
\'e\'en van de belangrijkste mogelijkheden tot uitbreidingen van het
Standaardmodel, in bijvoorbeeld de MSSM en 2HDM modellen. In het MSSM model
worden er b.v.~vijf Higgsbosonen voorspeld, waarvan twee geladen
(H$^{\pm}$) en drie neutraal zijn. \'E\'en van de neutrale deeltjes zou het SM
Higgs boson kunnen zijn 
($h$), \'e\'en een zwaar scalair deeltje en het derde een pseudoscalair
deeltje. De eigenschappen van het huidig waargenomen (SM) Higgsdeeltje moeten dus ook
gereproduceerd worden in elk van de modellen die uitbreidingen van het
Standaardmodel voorstellen; dit is de zogenaamde {\it alignment limit}.  

Het topquark (t) is \'e\'en van de zwaarste deeltjes van het SM en wordt
voornamelijk geproduceerd als onderdeel van een top-antitop quarkpaar. 
Het vervalt voornamelijk naar een
W boson en een bottomquark (b). Het W boson vervalt op zijn beurt verder ofwel
hadronisch naar twee deeltjesbundels, zogenaamde jets, of leptonisch naar een
lepton en het bijhorende 
neutrino. Het vervalkanaal van de W bosonen (leptonisch of hadronisch)
gecre\"eerd door de twee topquarks zorgt voor volgende klasseringen van
vervalmethoden: bi-leptonisch, semi-leptonisch of hadronisch. Omwille van de
grote massa van het topquark is er een grote koppeling tussen het topquark
en de Higgs bosonen. Dit zorgt ervoor dat een significante fractie van de
Higgsdeeltjes vervalt naar twee topquarks als dit kinematisch toegelaten
is. Dit vervalkanaal is voornamelijk belangrijk bij een $\mathcal{CP}$-oneven
deeltje  waarbij er niet kan gekoppeld worden aan de zwakke
ijkbosonen. In het 2HDM model echter verdwijnen de koppelingen van
een zwaar $\mathcal{CP}$-even deeltje in de "alignment limit", en in het MSSM
model zijn ze onderdrukt wanneer $m_A \gg m_Z$ (de
ontkoppelingslimiet). Omwille hiervan, maar ook om nog andere redenen, zouden
bepaalde theori\"en over de uitbreiding van het Standaardmodel (zogenaamde
{\it Beyond the Standard Model}, BSM theori\"en) zichtbaar kunnen worden bij
de productie van $t\bar t$ 
paren. In dit $t\bar t$ vervalkanaal is er echter een enorme achtergrond omwille
van andere SM productieprocessen, waardoor dit experimenteel
waar te nemen valt en de topquarks met hoge
precisie gereconstrueerd dienen te worden. Er werden al een aantal analyses
uitgevoerd waarbij gezocht werd naar 
resonanties in het $m_{t\bar t}$ spectrum, maar dat leverde tot nu toe nog geen positief
resultaat. Een belangrijk kenmerk van het $pp \rightarrow t\bar t$
proces in aanwezigheid van een zwaar Higgs boson is de interferentie met
de SM achtergrond. Omwille van een niet-triviale fase in de betrokken
"loop"-ge\"induceerde koppeling tussen het Higss boson en gluonen kan het
$m_{t\bar t}$ spectrum in plaats van de typische resonantie een piek-dip,
dip-piek of enkel een dip patroon vertonen. Dit spectrum wordt zelfs nog
verder uitgebreid wanneer BSM deeltes, zoals top squarks, in de loop voorkomen
of wanneer $\mathcal{CP}$ symmetriebreking in de Higgs sector
toegelaten wordt. Hierbij dient opgemerkt te worden dat zelfs in de limiet van
een smalle resonantie de impact van interferentie niet verwaarloosbaar is,
aangezien dit de hoogte van de piek be\"invloedt.  

De analyse die in deze thesis voorgesteld wordt ging op zoek naar een zwaar Higgsdeeltje dat
vervalt naar een topquarkpaar in het semileptonisch vervalkanaal. Er werd
hierbij gebruik gemaakt van de CMS 2016 proton-proton botsingsdataset met een totale 
luminositeit van 36\,fb$^{-1}$. De geselecteerde botsingsevenementen bevatten een
muon of een electron en ten minste vier jets. Hiervan
dienden er twee positief geselecteerd te zijn door het {\it b-tagging}
algoritme. De analyse beschouwde 
twee pure $\mathcal{CP}$ toestanden met massa's 400, 500, 600 and 750\,GeV, elk met
verschillende breedtes (2.5, 5, 10, 25, 50)\%. In totaal werden 40
($2\times 4 \times 5$) signaal datasets voor de resonantie en 40 voor de interferentie
geproduceerd. De interferentie met de QCD $t\bar t$ productie blijkt een
belangrijke rol te spelen en werd hier expliciet in rekening gebracht.  
 
De productie van de interferentie datasets vereiste dat de FORTRAN code gegenereerd
door \textsc{MadGraph5-\_a-mc\@nlo} % 1st hypen used for line adjustment
aangepast werd. $C_{\Phi tt}$ stelt hier de
koppeling van het zware Higgs boson met de topquarks voor, die gebruikt wordt door
het deel van de code dat het gekwadrateerde matrix element
$\abs{\mathcal{M}}^2$ evalueert. Deze code werd dan als volgt aangepast. Eerst wordt
$\abs{\mathcal{M}}^2$ ge\"evalueerd voor de nominale waarde van 
$C_{\Phi tt}$. Daarna wordt het teken van $C_{\Phi tt}$ omgekeerd, en wordt
$\abs{\mathcal{M}}^2$ nogmaals berekend, wat 
$\abs{\mathcal{M}(-C_{\Phi tt})}^2$ geeft. De effectieve koppeling van $\Phi$ met gluonen wordt
gecontroleerd door een onafhankelijke parameter in de code en wordt
daardoor niet be\"invloed door de aanpassing van $C_{\Phi tt}$. Het resultaat
is dat de interferentietermen in $\abs{\mathcal{M}}^2$ hun teken omwisselen,
terwijl de SM termen (onafhankelijk van de $\Phi$ koppelingen) en het
resonante BSM deel (evenredig met $C^2_{\Phi tt}$) onveranderd
blijven. Uiteindelijk wordt de waarde $(\mathcal{M}^{2}(C_{\Phi tt}) -
\mathcal{M}^{2}(-C_{\Phi tt})) / 2$ berekend, wat het originele kwadratisch
matrix element vervangt. Alle termen uitgezonderd de interferentietermen
vallen weg uit deze berekening.   

Het $t\bar t$ systeem werd gereconstrueerd door gebruik te maken van de
kinematische beperkingen te wijten aan de kennis van de massa van het topquark en
het W boson. De belangrijkste achtergrond is de $t \bar t$ productie, wat
gemodelleerd werd met \textsc{Powheg~v2} en \textsc{Pythia~8}. De overblijvende
(kleinere) achtergronden werden dan als volgt gemodelleerd: enkelvoudig top tw-kanaal
met  \textsc{Powheg} en \textsc{Pythia~8}, enkelvoudig top t-kanaal met
\textsc{Powheg~v2} en \textsc{Madspin}, enkelvoudig top s-kanaal en TTZ
\textsc{Amcatnlo} en $\textsc{Pythia~8}$, W jets en Drell-Yan met
\textsc{Madgraph} en $\textsc{Pythia~8}$, di-boson met $\textsc{Pythia~8}$,
TTZ en TTW jets met \textsc{Amcatnlo}, \textsc{Madspin} en
$\textsc{Pythia~8}$. 
 
De multijet achtergrond is onmogelijk om te modelleren via simulaties en werd
hier bepaald via technieken vertrekkende van de proton-proton dataset zelf. Het
aantal evenementen te wijten aan multijet achtergrond werd geschat 
met de zogenoemde ABCD methode. De methode werd toegepast voor de 
$M_{T}^W$ variabele en de relatieve leptonisolatie~$I$. De vorm van de
(tweedimensionale) verdeling van de onderzochte variabelen voor de multijet
achtergrond werd gemodelleerd aan de hand van de vorm die verkregen werd bij inversie
van het relatieve leptonisolatie criterium. De verwachtte bijdrage van achtergrond te wijten
aan prompte leptonen werd van de verdeling van de data afgetrokken en de
uiteindelijk resterende verdeling werd toegewezen aan de multijet achtergrond. 

Correcties aan de simulaties, voor {\it pile-up}, jet energieniveau
en resolutie, {\it b-tagging}, trigger en lepton efficienties en simulatiegewichten
werden doorgevoerd volgens de standaard CMS procedures. De top transversaal
moment $p_{T}$ correctie gebeurde met een empirische herschaling gebaseerd op de geobserveerde
distributies van topquarks. De empirische procedure werd voorts vergeleken met
de theoretische verwachting, waarbij de onzekerheid op de $p_T$ van het topquark 
in rekening werd gebracht. E\'en van de dominante onzekerheden in deze
analyse is te wijten aan de energiecalibratie van de jets. Deze onzekerheid
werd ge\"evalueerd door de 
multiplicatieve jet energiecorrectie te vari\"eren binnen zijn eigen
onzekerheden. JER data naar simulatie schaalfactoren werden ook gevarieerd
binnen hun onzekerheden, gelijktijdig voor alle jets. Onzekerheden gerelateerd
aan b-tagging werden ge\"evalueerd door de waarden van de schaalfactoren te
laten vari\"eren binnen hun onzekerheden. Ook onzekerheden te wijten aan
schaalfactoren van de trigger en de leptonidentificatie werden in rekening
gebracht. Voor de ge\"integreerde luminositeit werd een onzekerheid van
2.5~\% meegenomen.  
De normalisatie van de data-gedreven verdeling voor de multijet achtergrond werd
(conservatief) gekozen op +100~\% en -50~\%, onafhankelijk voor het muon- en
electronkanaal. Onzekerheden gerelateerd aan 
de renormalisatie- en factorisatieschaal ($\mu_{R}$ en $\mu_{F}$)
zijn verkregen via onafhankelijke variaties van de twee schalen met
een factor twee in beide richtingen. Onzekerheden te wijten aan de initi\"ele
en finale toestand straling (ISR, FSR) en aan het onderliggend evenement (underlying
event, UE) zijn bepaald via speciaal hiervoor gecre\"eerde sets. Topquark
massa onzekerheden werden verkregen via twee gesimuleerde datasets van $t\bar t$ met
top massa's $\text{m}_t$ van 169.5 and 175.5\,GeV. De $h_{\text{damp}}$
parameter in \textsc{POWHEG}, die de onderdrukking van
extra hoge $p_T$ jets bepaalt, werd gevarieerd van zijn nominale waarde
1.58$m_{t}$ tot 0.99$m_{t}$ en 2.24$m_{t}$. Ook de statistische onzekerheden
van de dominante $t\bar t$ achtergrond werden meegerekend. De
ge\"introduceerde variaties van de ISR en UE bleken statistisch niet
significant en werden hierom uiteindelijk niet verder in rekening genomen. De
PDF onzekerheden werden berekend van de set NNPDF3.0. Ze werden beschreven
door 100 alternatieve versies van de PDF (genoemd "MC replicas"), en werden
verder gebruikt als gewicht waarmee de gebeurtenissen kunnen gewogen
worden. 
  
Om de aanwezigheid van het hypothetische $\Phi$(H/A) deeltje te onderzoeken
werd een twee-dimensionale verdeling van de massa van het gereconstrueerde
$t\bar t$ systeem m$_{t\bar t}$ en een hoekvariabele die de spin van de
s-kanaal resonantie  weergeeft geconstrueerd. De eerste variabele m$_{t\bar
  t}$ geeft een benadering weer voor de massa van het zware Higgs deeltje. De
tweede variabele is $\abs{\cos\theta^*}$, waarbij $\theta^*$ de hoek is tussen de
impuls van het top quark dat leptonisch vervalt in het $t\bar{t}$ ruststelsel
en de impuls van het $t\bar{t}$~systeem in het lab stelsel. 
 
Voor de statistische analyse werd het zogenaamde {\it higgs combine tool}
gebruikt, gebaseerd op RooFit/RooStat. RooFit is een toolkit in het ROOT
softwarepakket dat verschillende types fits kan uitvoeren en 
ook toy monte-carlo datasets kan genereren gebaseerd op gebruiker-gedefineerde
modellen van PDFs. Het 
interferentie sjabloon, de 2D distributie van m$_{t\bar t}$ en $\cos\theta^*$,
heeft zowel positief als negatief gewogen evenementen. Aangezien het higgs
combine tool echter enkel met positieve gewichten kan werken, werd het
sjabloon in twee opgedeeld naargelang het teken van het gewicht van de
evenementen. Het teken van de negatieve gewichten werd vervolgens omgedraaid. De
twee sjablonen werden dan geschaald in het model met respectievelijk $g^2$ and
$-g^2$ (waarbij $g$ de koppeling schaalfactor is, en we voor de interferentie
een evenredigheid met $g^2$ verkrijgen), waarmee de uiteindelijk gewenste
verdeling geproduceerd werd. Gezien er geen indicatie voor een BSM signaal te
vinden was, werden voor diverse signaalhypotheses 
bovenlimieten voor de koppeling schaalfactoren bepaald in het semileptonisch kanaal. 
 
Om het interferentie-effect in rekening te brengen werd LO (leading order)
MADGRAPH gebruikt om de signaal datasets te genereren, waarna een herschaling
uitgevoerd werd naar een hogere orde (NNLO) werkzame doorsnede toe. De SUSHI
code samen met de 2HDMC calculator werden gebruikt om schaalfactoren voor het
resonantiedeel in een hMSSM context te berekenen. 
Voor het interferentiedeel werd de $\textsc{Top++}$
calculator gebruikt voor de berekening van de SM $t \bar t$ werkzame
doorsnede. De k-factor werd  berekend als de verhouding van de NNLO tot LO werkzame
doorsnede en werd daarna toegepast op de $\textsc{Madgraph}$ werkzame doorsnede voor het
signaal. Voor eenzelfde massapunt in deze analyse blijven de k-factoren
constant, d.w.z. onafhankelijk van alle beschouwde breedtes.  
De hier voorgestelde analyse is gevoelig voor hoge waarden van de $\Phi$ massa
en lage $\tan \beta$, een gebied dat bestreken wordt door het hMSSM waarin
rekening gehouden met het reeds ontdekte 125\,GeV Higgs boson. Aangezien het
onmogelijk is om het hele relevante ($m_A$, $\tan\beta$) gebied te bestrijken met een
gelimiteerde set van massa's en breedtes zoals in deze analyse gebruikt, werd
het $\textsc{Roomomentmorph}$ algoritme aangewend om te interpoleren tussen de
massa's en breedtes. Voorts is er ook een extrapolatie uitgevoerd naar
kleinere breedtes onder de {\it narrow-width} benadering.     

De resultaten voor de semileptonische finale toestand werden gecombineerd met
die van het dileptonische
kanaal, waarbij bovenlimieten bepaald werden voor de coupling modifier van de extra Higgs
bosonen met topquarks. De resultaten
werden binnen de hMSSM context ge\"interpreteerd in het $(m_A,
\tan\beta)$ vlak. Binnen de systematische onzekerheden zijn de resultaten
compatibel met de theoretische 
voorspellingen. Deze analyse is
gevoelig aan het lage $\tan\beta$ gebied en is dus complementair aan andere
analyses die ofwel gevoelig zijn voor hoge $\tan\beta$ ($2\tau$ finale
toestand) of voor massa's onder de $t\bar t$ productiedrempel\footnote{In
  het bijzonder analyses die zoeken naar bi-Higgs bosonen en pseudoscalairen die
  vervallen naar het 125\,GeV Higgs boson en een Z boson, alsook restricties
  komende van de geobserveerde koppelingen met het 125\,GeV Higgs boson.}.    

Tot nu toe werd er in deze analyse geen nieuwe fysica gevonden en zijn
de resultaten in overeenkomst met de voorspellingen van het
Standaardmodel. Een beter resultaat zou kunnen verkregen worden met meer 
statistiek in de data en met een meer preciese meting van de $t \bar t$
massa. De volledige Run-II dataset kan daarom de gevoeligheid in dit kanaal al
verhogen. Met de 300\,fb$^{-1}$ en 3\,ab$^{-1}$ aan data die voorzien wordt
gedurende respectievelijk Run-III en de HL-LHC zal het $gg\rightarrow
H\rightarrow t\bar t$ kanaal een centrale rol spelen in de zoektocht naar
zware Higss deeltjes. Omwille van zijn zelfde topologie is in deze analyse 
de SM $t\bar t$ productie de voornaamste achtergrond. Om de gevoeligheid van
deze analyse verder op te drijven dient deze 
achtergrond dan ook verder onderdrukt worden door middel van een verbeterde
signaal-tot-achtergrond discriminerende technieken. Dit kan in principe gebeuren door
gebruik te maken van MVA 
technieken waarbij de negatieve gewichten voor de interferentie in rekening gebracht worden. 
De k-factoren voor interferentie zijn momenteel berekend als
het geometrisch gemiddelde van de resonantie en de SM $t \bar t$ k-factoren. Bij
de volgende iteratie van deze analyse is het de bedoeling 
om de NLO k-factoren voor de interferentie te
berekenen op een manier waarbij de curves van de k-factoren het
interferentie-effect reeds in rekening brengen. Verschillende signaal curves zoals
een zuivere {\it dip}, {\it bump} of zelfs {\it nothingness} zullen ook helpen om dit kanaal
verder te onderzoeken.  

\clearpage{\pagestyle{empty}\cleardoublepage}

%\renewcommand*{\thesection}{\thechapter.\arabic{section}}       % reset again to chaptnum.sectnum
