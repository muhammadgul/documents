\graphicspath{{chapt_dutch/}{intro/}{chapt2/}{chapt3/}{chapt4/}{chapt5/}{chapt6/}{chapt7/}{chapt8/}{chapt9/}}

% Header
\renewcommand\evenpagerightmark{{\scshape\small Chapter 1}}
\renewcommand\oddpageleftmark{{\scshape\small Introduction}}

\renewcommand{\bibname}{References}

\hyphenation{}

\chapter[Introduction]%
{Introduction}\label{chap:intro}
Elementary particle physics aims to answers the most fundamental question asked by human kind – ``What is matter made of?” The journey of particle physics began from atomic theory that considers the atom to be the building block of matter. With the passage of time, scientists realised that the atom can be further disintegrated into smaller parts. The last two centuries have been the bright era of particle physics, where the knowledge pertaining to the fundamental building blocks of nature has developed rapidly, leading to changing ideas of what really is fundamental. The experimental investigation of the particles, carried out using detectors and accelerators, started from the discovery of x-rays, followed by electrons, protons, neutrons, and a zoo of other particles till the recent discovery of the Higgs boson. The detectors and accelerators used in the field of particle physics originated from simple technology like Geiger Counter, Cloud Chamber, etc., which evolved with time. Now we have the world's most powerful Large Hadron Collider (LHC) at the European Laboratory for Particle Physics (CERN) in Geneva. The LHC is going to be upgraded to High Luminosity LHC (HL-LHC) machine to increase the luminosity further by a factor of 10 in order to improve statistically marginal measurements.  
 
Theory and experiment run parallelly, and in the last few decades, a theory has emerged that describes all the known elementary particle interactions, except gravity, known as the Standard Model (SM) of particle physics. The SM is not the final word on the subject and has great scope for being extended, but at least we now have (for the first time) a full deck of cards to play with. The discovery of the Higgs boson completed the SM but left some unanswered questions. Why is the mass of the Higgs boson 125\,GeV? What is the origin of Dark Matter (DM) and Dark Energy? What is the origin of gravity? And so forth. Many beyond the SM (BSM) theories have been proposed to answer these questions and among them, supersymmetry (SUSY) is the leading possibility.

The LHC successfully began its operations in 2009 at low energy (1.18\,TeV per beam); on 30$^{\tiny{\text{th}}}$ March, 2010 at 13:06 CEST, two beams were collided at the centre of mass (CoM) energy, $\sqrt{s}$ = 7\,TeV (3.5\,TeV per beam), setting a record of high-energy $pp$ collision. The CoM energy further increased to 8~Te\, in April 2012 and the first new particle was discovered on 4$^{\tiny{\text{th}}}$ July, 2012 – the Higgs boson. The combined 7 and 8\,TeV data collected during the LHC Run did not point to BSM physics. After the first long shutdown, the machine began operation in May, 2015 by colliding protons at the record-breaking collision CoM energy of 13\,TeV (Run II), successfully delivering $\mathcal{L}\approx 80~\text{fb}^{-1}$ integrated luminosity in 2016–17. The high luminosity might provide answers to some of the questions through the discovery of new particle(s), applying limits on lower masses in the kinematic plane [mass, $\tan\beta$]. This will allow for a more precise test of the properties of the Higgs boson and top quark pair and the exploration of its rare decay modes, where deviations from the SM might be found. It will also open a window for searching for particles using high-mass ranges up to TeV scale in the heaviest SM particles’ final state.

The extensions of the SM, such as 2HDM and hMSSM predict more than one Higgs boson. There are five Higgs states – a light SM-like scalar $h$, a heavy scalar $H$, a pseudoscalar $A$, and two charged Higgs bosons $H^\pm$. The work presented in this thesis aims to search for heavy Higgs, $\mathcal{CP}$-even neutral H and $\mathcal{CP}$-odd neutral A, in the $t\bar t$ final state in the context of the so-called hMSSM that takes the existence of the 125\,GeV Higgs boson directly into account. The large mass of the top quark naturally leads to a strong coupling between it and the additional Higgs bosons. 

The High Luminosity Large Hadron Collider (HL-LHC) machine will induce higher background radiations as compared to the current operating conditions. It is important to study the performance and stability of the currently installed and future detectors in a high radiation environment. The CERN engineering and physics departments designed and built a dedicated Gamma Irradiation Facility (GIF++), where high-energy charged particle beams, mainly muon beams with a momentum up to 100\,GeV/c, are combined with a 13.9\,TBq $^{137}$Cs source. The GIF++ cumulates doses equivalent to the HL-LHC experimental conditions within a reasonable time frame. The CMS muon group installed many RPC detectors to test for the HL-LHC conditions. To operate the installed detectors, I designed and applied a dedicated detector control system project for the CMS RPCs at the GIF++ facility.

I began my PhD at the beginning of 2014 using 8\,TeV data for the charged Higgs searches, but we soon realized that the channel was not feasible owing to the small cross section. I moved towards the search for the neutral heavy Higgs in the $t\bar t$ final state and started from the generator level. I prepared the signal at 13\,TeV using fast simulation and then full simulation for the 2015 data set, processed all the data and background samples, applied all corrections to MC simulations and synchronized with other groups. LHC delivered about 2\,fb$^{-1}$ data in 2015, which was not sufficient for this search. We further extended the search to analyse the full 2016 data set. 

The analysis included in this thesis is a team effort, wherein I contributed with regard to the signal simulation, made synchronizations with other groups in our team, processed the entirety of the 2016 data and all simulated backgrounds, applied corrections to data and simulations, and compared data to backgrounds that showed good agreement. Further, my contribution involved making QCD templates for the final statistical analysis using data-driven techniques. I also applied top $p_{T}$ correction scale factors from theory to our simulated $t\bar t$ sample and compared them with the correction taken from real data. I calculated the k-factors to scale the signal both in semileptonic and dileptonic final states to higher order cross sections in order to interpret the results in the context of hMSSM.

The thesis is structured in the following way:
Chapter~\ref{chapt:2} describes the SM of physics, which mainly focuses on the top quark and SM Higgs boson. It further provides a detailed overview of Higgs physics in BSM, especially the heavy neutral Higgs in the $t\bar t$ final state with interference effects from the SM $t\bar t$. This provides the theoretical background for the search presented in the thesis. Chapter~\ref{chapt:lhc} briefly covers the LHC and CMS experiments. The High Luminosity LHC (HL-LHC), the Gamma Irradiation Facility (GIF++) setup designed to test the installed and new detectors for the HL-LHC conditions, the detector control system, based on the WinCC-OA framework, for the operation of the CMS Resistive Plate Chamber (RPC) installed in the GIF++ and the RPC performance results will be described in Chapter~\ref{chapt:4}. The Monte Carlo simulation, particle physics generators, object reconstruction in the CMS detector, and signal modelling will be discussed in Chapter~\ref{chapt:5}. Chapter~\ref{Research_str_meth} describes the full data analysis in the semileptonic state, including corrections to MC simulations, physics objects and event selection, search variables, multijet estimation from data driven techniques, experimental, theoretical and MC statistical uncertainties as well as the statistical analysis for semileptonic channel. The strategy used for combining semileptonic and dileptonic channels, masses and widths interpolation and extrapolation, and model independent and model dependent limits will be provided in Chapter~\ref{chapt:8}. The paper concludes with a summary and outlook in Chapter~\ref{summary_conclusion}.   

